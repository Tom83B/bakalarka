\documentclass[11pt]{article}
\usepackage[utf8]{inputenc}
\usepackage[IL2]{fontenc}
\usepackage[czech]{babel}
\usepackage[]{units}
\usepackage{fullpage}
\usepackage{pdflscape}
\usepackage{caption}
\usepackage{courier}
\usepackage{subcaption}
\usepackage{subfig}
\usepackage{amsmath}
\usepackage{amssymb}
\usepackage{graphicx}%,pxfonts} %matematicke prostredi ams
\usepackage{gensymb}

\def\D{\mathrm{d}}
\newcommand{\pder}[2][]{\frac{\partial#1}{\partial#2}}
\newcommand{\der}[2][]{\frac{\D#1}{\D#2}}

\begin{document}
\begin{align}
	\pder[T]{t} &= \pder[T]{U}\pder[U]{t}\\
	\pder[T]{U} &= \frac{1}{c_p(\rho V)} = \frac{1}{c_p\rho S(z)\D z}\\
\end{align}
kde $S(z)$ je obsah průřezu silotrubice. Změnu vnitřní energie počítám z úvahy, že úsek v hloubce $z$ ztrácí energii tím, že má nějaký výkon a naopak přijímá energii zespoda:
\begin{align}
	\pder[U]{t} &= -P(z)+P(z-\D z) = -F(z)S(z)+F(z-\D z)S(z-\D z)\\
	\pder[T]{t} &= -\frac{1}{c_p\rho S(z)}\pder[]{z}(F(z)S(z)) \label{eq:zmena_s_obsahem}
\end{align}
Vzhledem ke konstatnímu toku $\Phi$ lze definovat střední poloměr $\langle r\rangle$ dle rovnice (24) z Deinzera:
\begin{equation}
	\Phi=\pi\langle r \rangle^2 B \\
\end{equation}
a pak tedy platí:
\begin{equation}
	S = \frac{\Phi}{B}.
\end{equation}
Dosazením do (\ref{eq:zmena_s_obsahem}) získám
\begin{equation}
\begin{aligned}
	\pder[T]{t} &= -\frac{1}{c_p\rho}\pder[]{z}F(z)-\frac{F(z)\Phi}{c_p\rho S(z)}\pder[]{z}\left(\frac{1}{B(z)}\right)\\
				&= \frac{1}{c_p\rho}\left(F\pder[\ln B]{z}-\pder[F]{z}\right)
\end{aligned}
\end{equation}
\end{document}