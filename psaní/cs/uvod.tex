\chapter*{Úvod}
\addcontentsline{toc}{chapter}{Úvod}

Bipolární sluneční skvrny vzniknou, když magnetická silotrubice spojená s jádrem Slunce vystoupí nad povrch do fotosféry. Skvrny mají opačnou polaritu a nejprve jejich společný pohyb odpovídá tomu, že jsou navázány na magnetické jádro. Chování skvrn po několika dnech ale naznačuje, že se skvrny od tohoto jádra odpojily.

Ve článku \cite{dd} byl navržen mechanismus, jakým k tomuto jevu může dojít. Začne docházet k ochlazování skvrny, které prostupuje do hloubky. V důsledku tohoto dojde k poklesu tlaku plynu a následnému zesílení magnetického pole blízko povrchu. Konvektivními toky směrem dolů propagující ochlazování skvrny se v hloubce \unit[2-10]{Mm} pod povrchem střetnou s konvektivními toky směrem nahoru a dojde k nárůstu tlaku, který po několika dnech dosáhne hodnoty odpovídající tlaku v klidném Slunci v této hloubce a dojde zde k rozpadu magnetické silotrubice. Autoři provedli 1D simulaci tohoto mechanismu.

Cílem této práce je tuto simulaci zreprodukovat a umožnit tak model doplnit o fyzikálně realističtější podmínky; tedy sestavit modulovatelný kód, pro který bude snadné nahradit části jako stavová rovnice, opacitní tabulky nebo model konvekce.