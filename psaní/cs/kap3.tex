\chapter{Dynamická diskonexe}

Pozorování bipolárních slunečních skvrn nasvědčuje, že pár dní po jejich vynoření se silotrubice, která je tvoří, odpojí od magnetického jádra Slunce. Důvody jsou následující:
\begin{itemize}
	\item Velikost aktivních oblastí, kde skvrny vznikají, je mnohem rozsáhlejší, než je vzájemná vzdálenost dvou bipolárních skvrn. Kdyby tedy silotrubice byla dále napojena na magnetické jádro a vynořovala by se, očekávali bychom, že vzájemná vzdálenost skvrn bude odpovídat rozměrům dané aktivní oblasti. Vzdálenost skvrn ovšem v porovnání s velikostí aktivní oblasti zůstane nepatrná.
	\item Skvrny by se měly pohybovat směrem k pólům (tento jev nepozorujeme) \textbf{není mi jasné proč}
	\item Při vynořování na magnetickou silotrubici působí Coriolisova síla, která způsobí natočení spojnice skrvn vůči východo-západnímu směru. Poté, co vynořování skončí, Coriolisova síla přestává působit a očekáváme, že naklonění spojnice bude relaxovat zpět do jeho původní polohy; ani tento jev ovšem nepozorujeme.
\end{itemize}

Mechanismus navržený ve článku \cite{dd} se snaží vysvětlit z jakého důvodu a jak k odpojení od magnetického jádra.

Po vynoření sluneční skvrny se její povrch začne ochlazovat, což je doprovázeno tokem plynu směrem ke středu Slunce. Tím dojde k poklesu tlaku v povrchové vrstvě skvrny a zesílení magnetického pole v této oblasti. V hloubce mezi $\unit[2-10]{Mm}$ dojde vlivem toku plynu směřujícího do hloubky k nárůstu tlaku a po několika dnech tento tlak v určitém místě vyrovná tlak v okolí silotrubice a dojde k rozpojení silotrubice.

\section{Model dynamické diskonexe a model konvekce}

\subsection{Model konvekce}
Pozorování (viz. umbrální tečky) nasvědčuje tomu, že ve slunečních skvrnách nedochází pouze k radiativnímu přenosu energie, ale také ke konvektivnímu přenosu. Z rovnic hvězdného nitra vyplývá pro celkový tok a tok energie přenášený radiací
\begin{align}
	F_{rad}+F_{conv} &= \frac{4acG}{3}\frac{mT^4}{\kappa P r^2}\nabla_{rad} \label{eq:celk_tok}\\
	F_{rad} &= \frac{4acG}{3}\frac{mT^4}{\kappa P r^2}\nabla \label{eq:frad}
\end{align}
kde $\nabla_{rad}=\left(\frac{\D\ln T}{\D\ln p}\right)_e$ je logaritmický teplotní gradient potřebný pro to, aby energie byla přenášena pouze zářením, $\nabla$ je skutečný logaritmický teplotní gradient.

Skutečný teplotní gradient a tedy i vztah pro tok energie přenášené konvekcí záleží na použitém modelu konvekce. Pro astrofyzikální aplikace se velmi často používá model směšovacích délek (mixing length model).

Tento zjednodušující model předpokládá, že energie je přenášena elementem hmoty, který se po určitou dobu chová jako uzavřený systém o konstantním počtu částic a po určité uražené vzdálenosti (směšovací délce $l_m$) difunduje do okolí. Definujme pro veličinu $X$ operátor:
\begin{equation}
	DX = \left(\left(\frac{\D X}{\D z}\right)_e-\frac{\D X}{\D z}\right)\Delta r,
\end{equation}
kde $\left(\frac{\D X}{\D z}\right)_e$ je změna veličiny uvnitř elementu způsobená zvětšováním objemu pro zachování stejného tlaku, jako je v okolí (tedy $Dp=0$) a radiačními energetickými ztrátami. Potom bude po uražení vzdálenosti $\Delta r$ bude rozdíl mezi teplotou elementu a teplotou okolí
\begin{equation}
	D T = \left(\nabla_e-\nabla\right)\frac{T\Delta r}{H_p}, \label{eq:nne}
\end{equation}
kde $\nabla$ a $\nabla_e$ jsou logaritmické gradienty teploty vztažené k logaritmu tlaku, $H_p=\frac{\D z}{\D\ln p}$. Elementem o hustotě $\rho$, měrné tepelné kapacitě $c_p$ pohybujícím se rychlostí $v$ je čistě konvekcí přenesena energie $E_{conv}=\rho c_p D T$ a konvektivní energetický tok je tedy
\begin{equation}
	F_{conv} = \rho c_p v D T.
\end{equation}
Předpokládejme, že součin $vDT$ bude nabývat své střední hodnoty poté, co element urazí vzdálenost $\frac{l}{2}$. Element bude v důsledku vztlaku urychlován silou, jejíž hustota je
\begin{equation}
	k_r=-g \frac{D\rho}{\rho}=-g\pder[\ln\rho]{\ln T}\frac{DT}{T}=-g\delta\frac{DT}{T},
\end{equation}
kde
\begin{equation}
	\delta=\pder[\ln\rho]{\ln T}.
\end{equation}
Práce vztažená na jednotku hmotnosti po uražení vzdálenosti $l_m/2$ je tedy
\begin{equation}
	W = k_r\frac{l_m}{2}=\frac{1}{8}g\delta\frac{l_m^2}{H_p}(\nabla-\nabla_e).
\end{equation}
Za předpokladu, že práce je z poloviny vynaložena na odpor prostředí a z poloviny na kinetickou energii, můžeme střední rychlost $v$ určit ze vztahu
\begin{equation}
	\frac{1}{2}v^2=\frac{1}{2}W
\end{equation}
a tedy
\begin{equation}
	F_{conv} = \rho c_p T\sqrt{g\delta}\frac{l^2_m}{4\sqrt{2}}H_p^{-3/2}(\nabla-\nabla _e)^{3/2}. \label{eq:fconv1}
\end{equation}
Nebýt radiativního ochlazování elementu, platilo by $\nabla_e=\nabla_{ad}$. $\nabla_{ad}$ je tzv. adiabatický teplotní gradient, který lze lokálně spočítat pomocí vztahu
\begin{equation}
	\nabla_{ad}=\frac{\delta P}{c_P\rho T}
\end{equation}
Hustotu toku energie, kterou element vyzařuje lze v difúzní aproximaci vyjádřit jako
\begin{equation}
	f = \frac{-4ac}{3}\frac{T^3}{\kappa\rho}\frac{\D T}{\D n}.
\end{equation}
Normálový teplotní gradient $\frac{\D T}{\D n}$ můžeme pro kouli o průměru $d$ aproximovat jako $\frac{DT}{d/2}$ a tedy celkový zářivý výkon $\lambda$ takové koule je
\begin{align}
	\lambda &= Sf\\
			&= \frac{8acT^3}{3\kappa\rho}DT\frac{S}{d}.
\end{align}
Odtud již můžeme určit odchylku od změny při adiabatickém ději, jelikož $\frac{\D T}{\D r} = \frac{\D T}{\D Q}\frac{\D Q}{\D t}\frac{\D t}{\D r}$. Proto
\begin{align}
	\left(\frac{\D T}{\D r}\right)_e-\left(\frac{\D T}{\D r}\right)_{ad} &= -\frac{\lambda}{\rho V c_p v}\\
	\nabla _e-\nabla _{ad} &= \frac{8ac}{3\kappa}DT\frac{6}{d^2}.
\end{align}
Vydělením rovnicí (\ref{eq:nne}):
\begin{equation}
	\frac{\nabla _e-\nabla _{ad}}{\nabla-\nabla _e} = \frac{8acT^3}{\kappa\rho^2vc_p}\left(\frac{l_m}{d^2}\right).
\end{equation}
Element ovšem není dokonalá koule, proto se člen $\frac{8l_m}{d^2}$ nahrazuje $\frac{6}{l_m}$. Dále lze kombinací rovnic (\ref{eq:celk_tok}) a (\ref{eq:fconv1}) za použití $g=\frac{Gm}{r^2}$ vyjádřit
\begin{equation}
	F_{conv} = \rho c_p T\sqrt{g\delta}\frac{l^2_m}{4\sqrt{2}}H_p^{-3/2}\frac{8U}{9}(\nabla_{rad}-\nabla). \label{eq:fconv}
\end{equation}
V případě, že je známa závislost teploty na tlaku, lze z tohoto vzorce v každém bodě spočítat veličinu $F_{conv}$.

\subsection{Magnetohydrostatická rovnováha}
Dosazením magnetického pole v soběpodobném tvaru podle rovnic (\ref{eq:Bz}) a (\ref{eq:Br}) do rovnice pro magnetohydrostatickou rovnováhu (\ref{eq:mh_equilib}) získáme následující dvě rovnice pro magnetohydrostatickou rovnováhu:
\begin{align}
	0 &= -\pder[p]{r}+\frac{B_z}{\mu_0}\left(\pder[B_r]{z}-\pder[B_z]{r}\right)\\
	0 &= -\pder[p]{z}-\frac{B_r}{\mu_0}\left(\pder[B_r]{z}-\pder[B_z]{r}\right)-\rho g
\end{align}
První z těchto rovnic po integrování podle $r$ od $0$ do $\infty$ (pro konstantní $z$) přechází v rovnici
\begin{equation}
	\frac{\Phi}{2\pi}y\frac{\D^2y}{\D z^2}=y^4-2\mu_0(p_e-p_i), \label{eq:magstat}
\end{equation}
kde $y=\sqrt{B_0(z)}$, $\Phi$ je magnetický tok silotrubicí (nezávislý na hloubce), $p_e(z)=p(\infty,z)$ je vnější tlak a $p_i(z)=p(0,z)$ je tlak uvnitř silotrubice, který j dán podmínkou pro hydrostatickou rovnováhu:
\begin{equation}
	\frac{\D p_i}{\D z}=\rho_i g \label{eq:hydrostat}
\end{equation}

\subsection{Časový vývoj}
Jde o kvazistacionární model, předpokládá se tedy, že v každém okamžiku je soustava v magnetohydrostatické rovnováze, která je určena rovnicemi (\ref{eq:magstat}) a (\ref{eq:hydrostat}). V důsledku přenosu energie se ovšem v průběhu mění teplota a v důsledku toku plynu směrem od středu Slunce na povrch se bude měnit také tlak.

Pro změnu teploty $T_i$ v důsledku přenosu energie platí
\begin{equation}
	\pder[T_i]{t}=-\frac{1}{\rho_i c_p} \pder[]{z}(F_{rad}+F_{conv}),
\end{equation}
kde $F_{rad}$ a $F_{conv}$ jsou dány vztahy (\ref{eq:frad}) a (\ref{eq:fconv}).

V případě tenké silotrubice lze celkovou hmotu v silotrubici odhadnout vztahem
\begin{equation}
	m = \Phi\int\limits_0^{z_0}\frac{\rho_i}{B}\,\D z.
\end{equation}
Pokud na hranici $z_0$ vtéká hmota rychlostí $v_0$, lze tedy tuto změnu vyjádřit jako
\begin{equation}
	\Delta m=\Phi\frac{\rho_i(z_0)}{B(z_0)}v_0\Delta t.
\end{equation}
Změna tlaku v hloubce $z_0$ za časový úsek $\Delta t$ tedy musí být taková, aby bylo splněno
\begin{equation}
	m[p_i(z_0)+\Delta p_i(z_0)]=m[p_i(z_0)]+\Phi\frac{\rho_i(z_0)}{B(z_0)}v_0\Delta t
\end{equation}