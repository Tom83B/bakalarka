\chapter{Simulace dynamické diskonexe}

\section{Model klidného Slunce}
Pro umožnění spočtení magnetického pole podle rovnice (\ref{eq:magstat}) je potřeba znát tlak plynu jako funkci hloubky v klidném Slunci $p_e(z)$. Za tímto účelem lze použít některý dostupných modelů. Výpočet byl ovšem v rámci práce pro přípovrchvé vrstvy proveden pro ověření správnosti použitého modelu konvekce.

\subsection{Rovnice přenosu energie}
V difúzní aproximaci lze logaritmický teplotní zářivý gradient $\nabla_{rad}$ vypočítat podle vztahu
\begin{equation}
	\frac{3\kappa P L_R}{16\pi a c G M_R T^4},
\end{equation}
kde $L_R$ je celkový zářivý výkon Slunce ve vzdálenosti $R$ od jeho středu, $M_R$ je hmotnost části Slunce obalené slupkou o poloměru $R$. V přípovrchových vrstvách, kterými se v práci zabýváme, dochází již pouze k naprosto zanedbatelnému počtu jaderných reakcí, při kterých by byla produkována energie a které by ovlivnily celkový zářivý výkon. Proto byla pro výpočet tato hodnota nahrazena hodnotou $L^*$, tedy zářivým výkonem celého Slunce. Tento gradient určuje, jak se mění teplota v závislosti na tlaku, pokud uvažujeme pouze zářivý přenos energie.

Označme logaritmický teplotní gradient získaný z modelu konvekce jako $\nabla_{conv}$. Jelikož bude vždy automaticky zvolen efektivnější způsob přenosu energie, můžeme pro skutečný teplotní gradient $\nabla$ psát
\begin{equation}
	\nabla = \min(\nabla_{ad},\nabla_{conv}).
\end{equation}